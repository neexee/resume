\documentclass[final]{letter}
\usepackage{nopageno}
\usepackage{rotating}
\usepackage{tabularx}
\usepackage{fix-cm}
\usepackage{fullpage}
\usepackage[utf8]{inputenc}
\usepackage[russian]{babel}
\usepackage{lmodern}
\usepackage[T1,T2A]{fontenc}

\usepackage{fontspec}
\setmainfont{Liberation Sans}

\usepackage[babel=true]{microtype}

\usepackage{hyperref}
\usepackage{fancyhdr}
\usepackage{xcolor}

\addtolength{\textwidth}{1.0in}
\addtolength{\hoffset}{-.6in}
\addtolength{\textheight}{1.6in}
\addtolength{\voffset}{-.7in}
\hypersetup{
    colorlinks,
    linkcolor={red!50!black},
    citecolor={blue!50!black},
    urlcolor={blue!80!black}
}
\begin{document}
\begin{center}

% Title bar
{\fontsize{25}{40}\selectfont\bf{Мачульскис Сергей}}
  {\hfill
    \begin{tabular}{c}
        \href{mailto:machulskissv@gmail.com}{machulskissv@gmail.com}\\
        +7 913 2066438\\
     \end{tabular}
  }
\rule{.98\textwidth}{1pt}

% Need verticle space after hbar
\addvspace{.1cm}

\end{center}

{\bf Работа}
\begin{itemize}
\item с 2015 -- \href{https://2gis.com}{2ГИС}. C++, Python, Go, Scala. Делаю CDN с данными городов для мобильных приложений, приём и обработку пользовательской статистики, геопозиционирование пользователей по Wi-Fi и сотовым вышкам. Делал приём и хранение информации о пробках на дорогах и бэкенд поиска для 2ГИС-онлайн.
\item с 2017 -- \href{http://fit.nsu.ru/}{ФИТ НГУ}, преподаю основы Unix.
\end{itemize}

{\bf Образование}
\begin{itemize}
\item 2009 -- 2015. \href{https://www.nsu.ru/n/}{НГУ}, факультет информационных технологий.
\end{itemize}

{\bf Языки}
\begin{itemize}
\item \textit{Python}. Основной язык. Писал нагруженные бэкенды с asyncio внутри, большое приложение управления CDN, а также десятки небольших утилиток. Люблю писать тесты, потому что язык их очень просит. Скептически отношусь к ORM, но sqlalchemy мне нравится.
\item \textit{C++}. На уровне понимания «Effective Modern C++». Люблю язык за безграничные возможности. Пример кода можно посмотреть \href{https://github.com/neexee/review}{на github}.
\item \textit{Go}. Язык хороший, если нужен DSL для сетевого приложения или небольшая быстрая CLI-утилитка. Действительно большие проекты на нём не писал, мне кажется, Go не для них. Написал на нём несколько сервисов. Получил огромное удовольствие от переписывания Python-legacy на Go.
\item \textit{Scala}. Люблю Apache Kafka, но библиотеки для Go и C++ не поспевают за её развитием. Scala для Kafka нативная, поэтому качественные сервисы было логично писать на ней. Немного разбираюсь в функциональном программировании, это часто помогает писать хороший код в не совсем функциональных языках.
\end{itemize}

{\bf С чем приятно иметь дело}
\begin{itemize}
\item \textit{GNU/Linux}. Ни разу не подводил меня ни как пользователя (я бывший гентушник со стажем), ни как программиста на C++ и Go. Буду преподавать курс операционных систем в этом году.
\item \textit{Распределённые системы}. Reliability, Maintainability, Scalability! Понимание на уровне «Designing Data-Intensive Applications». Могу объяснить, что происходит в tcpdump + wireshark.
\item \textit{Документация}. Стремлюсь объяснять сложные вещи простыми словами. Если получается, очень радуюсь. Пока пытаюсь научиться писать через \href{https://www.goodreads.com/review/list/43338630-sergey-machulskis?shelf=professional}{умение читать}.
\item \textit{Базы данных}. Хорошая БД снимает головную боль, а не создаёт её. Примеры хороших БД на мой вкус -- \textit{Redis}, \textit{Tarantool} и \textit{ClickHouse}.
\item \textit{Git}. Знаю десяток команд. Написал \href{https://github.com/neexee/review/tree/master/components/Git}{RAII-обёртку} над libgit2.
\item \textit{LXD}. Раньше пользовался Docker, но мне больше по душе контейнеры со своим syslog и init, чем single app container.
\item \textit{Однокнопочный деплой}. Пользовался Ansible, он неплох. С удовольствием попробую что-нибудь ещё.
\end{itemize}

\vfill
\begin{center}
\underline{~~~~~~~~~~~~~~~~~}

08.06.2018
\end{center}

\end{document}
