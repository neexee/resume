\documentclass[final]{letter}
\usepackage{nopageno}
\usepackage{rotating}
\usepackage{tabularx}
\usepackage{fix-cm}
\usepackage{fullpage}
\usepackage[utf8]{inputenc}
\usepackage[russian]{babel}
\usepackage{lmodern}
\usepackage[T1,T2A]{fontenc}

\usepackage{fontspec}
\setmainfont{Liberation Sans}

\usepackage[babel=true]{microtype}

\usepackage{hyperref}
\usepackage{fancyhdr}
\usepackage{xcolor}

\addtolength{\textwidth}{1.0in}
\addtolength{\hoffset}{-.6in}
\addtolength{\textheight}{1.6in}
\addtolength{\voffset}{-.7in}
\hypersetup{
    colorlinks,
    linkcolor={red!50!black},
    citecolor={blue!50!black},
    urlcolor={blue!80!black}
}
\begin{document}
\begin{center}

% Title bar
{\fontsize{25}{40}\selectfont\bf{Machulskis Sergey}}
  {\hfill
    \begin{tabular}{c}
        \href{mailto:machulskissv@gmail.com}{machulskissv@gmail.com}\\
        Novosibirsk \\
        +7 913 2066438\\
     \end{tabular}
  }
\rule{.98\textwidth}{1pt}

% Need verticle space after hbar
\addvspace{.1cm}

\end{center}

{\bf Job}
\begin{itemize}
  \item since 2015 -- \href{https://2gis.com}{2GIS}
  \begin{itemize}
  \item Maintained a CDN delivering data to 20 millions of users.
    Converted CDN-controlling CLI for humans to a well-structured RESTful API for robots.
    Spent a fair amount of time improving Python code quality and performance.
  \item Built fault-tolerant cross datacenter pipeline based on Kafka.
    It replaced the old single datacenter ZeroMQ-based pipeline. Made the transition seamless for consumers.
  \item Designed JSON-based RESTful API to replace Apache Thrift API for a search backend of the online version of 2GIS.
    Also did some heavy lifting using templates in C++ there.
  \item Brought a new life to a dying Python-based geopositioning service. It was slow, hard to scale and fault intolerant.
    Rewrote and stabilized it using asyncio, it could handle 3000 rps.
    Rewrote it in Go, it could handle these 3000 rps without any special requiremens to a hardware.
  \item Wrote a hundred of documents. It's design decisions, analytics, API and service overviews, tutorials, SLAs and postmortems.
  \item Created and executed several long-term team roadmaps. Decomposed, estimated and prioritized hundreds of tasks.
  Actively participated in sprint planning and process improvement.
  \item Coordinated cross-team feature development and releases.
  \end{itemize}

  \item since 2017 -- \href{http://fit.nsu.ru/}{Novosibirsk State University}
    \begin{itemize}
      \item Taught courses on operating systems (system calls), parallel programming (pthreads) and computing history.
    \end{itemize}
  \end{itemize}

{\bf Education}
\begin{itemize}
  \item 2009 -- 2015. \href{https://www.nsu.ru/n/}{Novosibirsk State University}. Master's degree in Information Technology.
\end{itemize}

{\bf Languages}
\begin{itemize}
\item \textit{Python}. I've been using it almost every day for the past 5 years.
  Wrote hundreds of tests using pytest. Rewrote thousands of lines to satisfy mypy.
  I value well-designed libraries like aiohttp, SQLAlchemy, Requests and Keras.
\item \textit{C++}. Started the career using this language (С++14). I rarely see it even in reviews nowadays, my skills are rusty.
  There's an example of \href{https://github.com/neexee/review}{my code}.
\item \textit{Go}. Used it as a DSL for network applications. Some services were successful, some were not.
   Rewrote the old Python service in it, it was very satisfying.
\item \textit{Scala}. Used it not because I like complex languages but because it allows to write high-quality Kafka-based services.
  Go and C++ connectors for Kafka are not as good as their JVM-based counterparts.
  Learned some facts about functional programming using it. It helps to write better code even in imperative languages.
\end{itemize}

{\bf Interests}
\begin{itemize}
  \item \textit{Learning}. Lifelong learner. The \href{https://www.goodreads.com/review/list/43338630-sergey-machulskis?shelf=professional}{list of books} I've read.
  \item \textit{Strategy}. I like to come up with a great plan and analyze it down to small, invisible details. Usually it works out well.
  \item \textit{GNU/Linux}. Happy Gentoo/Ubuntu user for a decade. I know enough about UNIX-like OSs to teach it at the university.
  \item \textit{Distributed systems}. Defended a thesis on this topic, it's an old passion. I appreciate the motto ``Reliability, Maintainability, Scalability!''.
     Can explain what's happening in tcpdump + wireshark, Zookeeper and WALs.
  \item \textit{Documentation}. I'm trying hard to explain complex things in a simple manner. Did deep research on this topic.
  \item \textit{UX}. I consider myself as a backend programmer just because I don't like GUIs. But I care deeply about UX.
  \item \textit{Databases}. There's a list I'm happy to discuss and use: \textit{Redis}, \textit{Tarantool} \textit{PostgreSQL} and \textit{ClickHouse}.
  \item \textit{Source control}. I've been preparing releases with dozens of branches for years.
     Wrote \href{https://github.com/neexee/review/tree/master/components/Git}{a wrapper} around libgit2.
  \item \textit{Communication}. Solved a lot of ``technical'' prolems just by a productive conversation.
    I know some common cognitive biases. I strive to be a good listener, not a talker.
\end{itemize}

\end{document}
