\documentclass[final]{letter}
\usepackage{nopageno}
\usepackage{rotating}
\usepackage{tabularx}
\usepackage{fix-cm}
\usepackage{fullpage}
\usepackage[russian]{babel}
\usepackage[T2A]{fontenc}
\usepackage[utf8]{inputenc}
\usepackage{hyperref}
\usepackage{fancyhdr}
\usepackage{xcolor}

\addtolength{\textwidth}{1.0in}
\addtolength{\hoffset}{-.6in}
\addtolength{\textheight}{1.6in}
\addtolength{\voffset}{-.7in}
\hypersetup{
    colorlinks,
    linkcolor={red!50!black},
    citecolor={blue!50!black},
    urlcolor={blue!80!black}
}
\begin{document}
\begin{center}

% Title bar
{\fontsize{25}{40}\selectfont\bf{Мачульскис Сергей}}
  {\hfill
    \begin{tabular}{c}
        \href{mailto:machulskissv@gmail.com}{machulskissv@gmail.com}\\
        +7 913 2066438\\
     \end{tabular}
  }
\rule{.98\textwidth}{1pt}

% Need verticle space after hbar
\addvspace{.1cm}

\end{center}

{\bf Образование}
\begin{itemize}
\item 2009-2013. Бакалавриат ФИТ НГУ. Выпускная работа была посвящена кластеризации больших графов.
\item 2013-2015. Магистратура ФИТ НГУ. Занимался разработкой статического анализатора кода специализированного языка, предназначенного для реализации больших численных моделей на суперкомпьютерах.
\end{itemize}

{\bf Знания}
\begin {itemize}
\item \textit{C++.} Знаком с языком более пяти лет, каждый год улучшал свои знания о нём. Писал на C++ магистерскую работу(статический анализатор кода). Для удовольствия писал 
бота для XMPP-конференций(\href{https://github.com/neexee/neexee/blob/master/src/module/smart/smartmodule.cpp}{пример старого кода}) и виртуальную машину, исполняющую байт-код самодельного языка программирования.
Недавно начал делать библиотеку персистентных структур данных (\href{https://github.com/neexee/persist}{код}).
При выборе литературы по C++ отдавал предпочтение таким книгам, как <<Effective Modern C++>>, <<Effective STL>>.
\item \textit{Некоторые другие языки программирования.}
\begin{itemize}
\item \textit{Java}. На уровне <<Thinking in Java>>. Использовал для написания бакалаврской работы.
\item \textit{Python} на уровне прочитавшего <<Dive Into Python>>.
Изучал по собственной инициативе, нужен был для того, чтобы периодически писать короткие скрипты, например, для администраторских задач. Самый крупный проект -- распределенный интерпретатор
самодельного языка программирования.
\item Имею представление о \textit{C\#}, \textit{Perl}, \textit{Ruby}, \textit{Clojure}. Для Clojure пользовался библиотекой, реализующей модель акторов на легковесных тредах, очень понравилось.
\item \textit{SQL.} MySQL, Oracle SQL. Пользовался такими библиотеками, как JDBC.
\end{itemize}
\item \textit{Параллельные вычисления.} Для C/C++ знаком с набором инструментов, предназначенных для параллельного программирования в общей и распределённой памяти:
OpenMP, MPI, Pthreads, средства C++11, немного Boost.Thread. Знаю сложности разработки и проблемы параллельных программ.
\item \textit{Паттерны проектирования} и некоторые идиомы, важные для C++, например, Pimpl, RAII.
\item \textit{Системы контроля версий.} Несколько лет пользовался Git, немного SVN.
\item \textit{Системы автоматизации сборки.} Autotools, cmake.

\end{itemize}

{\bf Профессиональные интересы}
\begin {itemize}
\item \textit{UNIX-like системы.} Опытный пользователь GNU/Linux, особенно Gentoo Linux. Интересовался устройством ядра, прочёл <<Linux Kernel Development>>.
\item \textit{Параллельное программирование.} Существует множество интересных областей, которые я еще не затронул, например, распараллеливание на видеокартах, Erlang, lock-free структуры данных и многое другое.
\end{itemize}

\vfill
\begin{center}
\underline{~~~~~~~~~~~~~~~~~}

19.07.2015
\end{center}

\end{document}
